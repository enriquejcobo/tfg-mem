%---------------------------------------------------------------------
%
%                      resumen.tex
%
%---------------------------------------------------------------------
%
% Contiene el cap�tulo del resumen en ingl�s.
%
% Se crea como un cap�tulo sin numeraci�n.
%
%---------------------------------------------------------------------

\chapter{Abstract}
\cabeceraEspecial{Abstract}

A sensor network is formed by distributed sensors, whose task is monitor some parameters of its environment, exchanging information in order to obtain those that is relevant, and actuate as a consequence. This networks, in general, are connected wirelessly, which gives them more versatility and flexibility. However, this fact causes some challenges, such as energetic autonomy, as a result of being powered through batteries, and communications fiability.

\vspace{.1cm}

The implementations include fields such as home (domotics, aid services), industry (smart management of buildings), security and environment (measuring of parameters over burned down lands, earthquakes monitor), and others.

\vspace{.1cm}

In order to improve both communications fiability and energetic autonomy of WSN (Wireless Sensor Networks), cognitive properties are introduced in its platforms, which give it capabilities of value the spectrum state and, as a result, modify its communication parameters such in adaptive way. This networks are called CWSN (Cognitive Wireless Sensor Networks).

\vspace{.1cm}

Demarcated on the Technical University of Madrid, and particularly in the Integrated Systems Laboratory of the Electronic Engineering Department, this networks are one of the most important research lines, with proyects such as the deployment of a test-bed based on the node developed in the laboratory, cNGD (Cognitive Next Generation Device).

\vspace{.1cm}

This proyect will be focused in the design, implementation and test of a board --as known as shield-- for cNGD platform, in which both sensors and actuators are included, so actuators will intervene over the environment as a result of sensors measure. In addition, software needs will be resolved, and an application showing its operation will be developed. 

\vspace{.5cm}

\begin{table}[h!]
\Large
\scalebox{0.8}{
\begin{tabular}{ l l }
\textbf{\emph{KEY WORDS}}:	& \emph{cognitive network}, \emph{wireless cognitive network}, \\ 
 				& \emph{actuators}, \emph{shield}, \emph{device},\\
				& \emph{cognitive wireless sensor network}.
\end{tabular}}
\end{table}

\endinput
% Variable local para emacs, para  que encuentre el fichero maestro de
% compilaci�n y funcionen mejor algunas teclas r�pidas de AucTeX
%%%
%%% Local Variables:
%%% mode: latex
%%% TeX-master: "../Tesis.tex"
%%% End:
