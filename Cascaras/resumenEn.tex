%---------------------------------------------------------------------
%
%                      resumen.tex
%
%---------------------------------------------------------------------
%
% Contiene el cap�tulo del resumen en ingl�s.
%
% Se crea como un cap�tulo sin numeraci�n.
%
%---------------------------------------------------------------------

\chapter{Abstract}
\cabeceraEspecial{Abstract}

A sensor network is formed by distributed sensors whose task is to monitor some parameters of their environment, exchanging information in order to obtain those that are relevant, and act consequently. These networks, in general, are connected wirelessly, which gives them more versatility and flexibility. However, this fact causes some challenges, such as energy autonomy, due to being powered by batteries, and communications reliability.

\vspace{.5cm}

In order to improve both the communications reliability and the energy efficiency of WSNs (Wireless Sensor Networks), cognitive properties are introduced in these platforms. They give them the ability to sense the spectrum state and, as a result, modify their communication parameters in an adaptive way. These networks are called CWSNs (Cognitive Wireless Sensor Networks).

\vspace{.5cm}

In the Laboratorio de Sistemas Integrados (LSI) of the Departamento de Ingenier�a Electr�nica (DIE), these networks are one of the most important research lines, with projects such as the deployment of a test-bed based on the cNGD (Cognitive Next Generation Device), a node developed in the laboratory.

\vspace{.5cm}

This project is focused on the design, implementation and test of a board --known as shield-- for the cNGD. This board will include both sensors that can measure the environment, such as its temperature, acceleration, presence or luminosity; and actuators that will act upon it, modifying these parameters in a direct way or making a call in order to request a human intervention.

\vspace{.5cm}

In addition, software needs will be resolved by developing functions that allow the government of those sensors and actuators, used in combination with the cNGD platform. Finally, a demo application will be developed in order to show the system features.

\vspace{1cm}

\begin{table}[h!]
\Large
\scalebox{0.8}{
\begin{tabular}{ l l }
\textbf{\emph{KEY WORDS}}:	& \emph{cognitive wireless sensor network}, \\ 
 				& \emph{sensors and actuators}, \emph{shield}, \\
				& \emph{cognitive network}, \emph{wireless sensor network}, \\
 				& \emph{hardware design}.
\end{tabular}}
\end{table}

\endinput
% Variable local para emacs, para  que encuentre el fichero maestro de
% compilaci�n y funcionen mejor algunas teclas r�pidas de AucTeX
%%%
%%% Local Variables:
%%% mode: latex
%%% TeX-master: "../Tesis.tex"
%%% End:
