%---------------------------------------------------------------------
%
%                      resumen.tex
%
%---------------------------------------------------------------------
%
% Contiene el capítulo del resumen en inglés.
%
% Se crea como un capítulo sin numeración.
%
%---------------------------------------------------------------------

\chapter{Abstract}
\cabeceraEspecial{Abstract}

A sensor network is formed by distributed sensors, whose task is monitor some parameters of its environment, exchanging information in order to obtain those that is relevant, and actuate consequently. This networks, in general, are connected wirelessly, which gives them more versatility and flexibility. However, this fact causes some challenges, such as energetic autonomy, as a result of being powered through batteries, and communications fiability.

\vspace{.5cm}

In order to improve both communications fiability and energetic autonomy of WSN (Wireless Sensor Networks), cognitive properties are introduced in this platforms, which give it capabilities of value the spectrum state and, as a result, modify its communication parameters such in an adaptive way. This networks are called CWSN (Cognitive Wireless Sensor Networks).

\vspace{.5cm}

In the Integrated Systems Laboratory of the Electronic Engineering Department, this networks are one of the most important research lines, with projects such as the deployment of a test-bed based on the cNGD (Cognitive Next Generation Device), a node developed in the laboratory.

\vspace{.5cm}

This project will be focused in the design, implementation and test of a board --as known as shield-- for the cNGD. This board will include both sensors that could measure the environment, such as its temperature, luminosity, presence or acceleration over the board; and actuators that will intervene on it, modifying this parameters in a direct way or making a call in order to produce an human intervention.

\vspace{.5cm}

In addition, software needs will be resolved, by developing functions that allow the government of those sensors and actuators, used in combination with cNGD platform. Finally, a demo application will be defined in order to show the system features.

\vspace{1cm}

\begin{table}[h!]
\Large
\scalebox{0.8}{
\begin{tabular}{ l l }
\textbf{\emph{KEY WORDS}}:	& \emph{cognitive wireless sensor network}, \\ 
 				& \emph{sensors and actuators}, \emph{shield}, \\
				& \emph{cognitive network}, \emph{wireless cognitive network}, \\
 				& \emph{hardware design}.
\end{tabular}}
\end{table}

\endinput
% Variable local para emacs, para  que encuentre el fichero maestro de
% compilación y funcionen mejor algunas teclas rápidas de AucTeX
%%%
%%% Local Variables:
%%% mode: latex
%%% TeX-master: "../Tesis.tex"
%%% End:
