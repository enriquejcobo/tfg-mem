%---------------------------------------------------------------------
%
%                      resumen.tex
%
%---------------------------------------------------------------------
%
% Contiene el cap�tulo del resumen en espa�ol.
%
% Se crea como un cap�tulo sin numeraci�n.
%
%---------------------------------------------------------------------

\chapter{Resumen}
\cabeceraEspecial{Resumen}

Una red de sensores es aquella formada por sensores distribuidos cuya misi�n es la de monitorizar diferentes par�metros del entorno, intercambiando informaci�n para conocer la parte del mismo que nos interese y actuar en consecuencia. Estas redes, por lo general, se conectan de forma inal�mbrica, lo que les aporta mayor versatilidad y flexibilidad, si bien esto plantea retos de autonom�a energ�tica, al ser alimentadas por medio de bater�as, y de fiabilidad en las comunicaciones.

\vspace{.5cm}

Para mejorar tanto la fiabilidad de las comunicaciones como el consumo energ�tico de las redes de sensores inal�mbricas, introducimos propiedades cognitivas, de forma que los nodos sean capaces de apreciar el estado del espectro  y, como resultado, modificar sus par�metros de comunicaci�n de forma adaptativa. Estas redes se denominan CWSN (Cognitive Wireless Sensor Network).

\vspace{.5cm}

El trabajo se centrar� en el dise�o, implementaci�n y prueba funcional de una placa de expansi�n para un nodo de una red de sensores inal�mbrica cognitiva, denominado cNGD (Cognitive Next Generation Device). En esta placa se incluir�n tanto sensores que permitan realizar medidas del medio, como su temperatura, luminosidad, presencia de personas o aceleraci�n sobre la misma; como actuadores que, como consecuencia de dichas medidas, intervendr�n sobre �l, modificando bien par�metros directos como la temperatura, o avisando para que se produzca una acci�n humana.

\vspace{.5cm}

Asimismo, se realizar� el desarrollo del software necesario para su funcionamiento dentro del nodo, mediante la creaci�n de funciones que permitan gobernar dichos sensores y actuadores en combinaci�n con el cNGD, de forma que se permita la creaci�n sencilla de aplicaciones conjuntas. Finalmente, se definir� un ejemplo de aplicaci�n que muestre el funcionamiento del sistema. 

\vspace{1cm}

\begin{table}[h!]
\Large
\scalebox{0.8}{
\begin{tabular}{ l l }
\textbf{\emph{PALABRAS CLAVE}}:	& \emph{redes de sensores inal�mbricas cognitivas}, \\ 
 				& \emph{sensores y actuadores}, \emph{placa de expansi�n}, \\
				& \emph{redes cognitivas}, \emph{redes de sensores inal�mbricas}.
\end{tabular}}
\end{table}

\endinput


% Variable local para emacs, para  que encuentre el fichero maestro de
% compilaci�n y funcionen mejor algunas teclas r�pidas de AucTeX
%%%
%%% Local Variables:
%%% mode: latex
%%% TeX-master: "../Tesis.tex"
%%% End:
