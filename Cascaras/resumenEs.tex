%---------------------------------------------------------------------
%
%                      resumen.tex
%
%---------------------------------------------------------------------
%
% Contiene el cap�tulo del resumen en espa�ol.
%
% Se crea como un cap�tulo sin numeraci�n.
%
%---------------------------------------------------------------------

\chapter{Resumen}
\cabeceraEspecial{Resumen}

Una red de sensores es aquella formada por sensores distribuidos cuya misi�n es la de monitorizar diferentes par�metros del entorno, intercambiando informaci�n para conocer la parte del mismo que nos interese y actuar en consecuencia. Estas redes, por lo general, se conectan de forma inal�mbrica, lo que les aporta mayor versatilidad y flexibilidad, si bien esto plantea retos de autonom�a energ�tica, al ser alimentadas por medio de bater�as, y de fiabilidad en las comunicaciones.

\vspace{.1cm}

Las aplicaciones abarcan �mbitos como el hogar (dom�tica, servicios de teleasistencia), la industria (gesti�n eficiente de edificios), la seguridad y el medio ambiente (medici�n de par�metros sobre terrenos incendiados, monitorizaci�n de terremotos), entre otros.

\vspace{.1cm}

Para mejorar tanto la fiabilidad de las comunicaciones como el consumo energ�tico de las redes de sensores inal�mbricas, introducimos propiedades cognitivas, de forma que los nodos sean capaces de apreciar el estado del espectro  y, como resultado, modificar sus par�metros de comunicaci�n de forma adaptativa. Estas redes se denominan CWSN (Cognitive Wireless Sensor Network).

\vspace{.1cm}

En el contexto de la Universidad Polit�cnica, y m�s particularmente dentro del Laboratorio de Sistemas Integrados del Departamento de Ingenier�a Electr�nica, estas redes son uno de los principales objetos de investigaci�n, con proyectos como la puesta en marcha de un banco de pruebas basado en el nodo desarrollado en el laboratorio, cNGD (Cognitive Next Generation Device).

\vspace{.1cm}

El trabajo se centrar� en el dise�o, implementaci�n y prueba de una placa de expansi�n para el cNGD, en el que se incluir�n tanto sensores, que permitan realizar medidas del medio, como actuadores, que, como consecuencia de dichas medidas, intervendr�n sobre �l. Asimismo, se realizar� el desarrollo del software necesario para su funcionamiento dentro del nodo y se definir� una aplicaci�n que muestre el funcionamiento del sistema en su conjunto. 

\vspace{.5cm}

\begin{table}[h!]
\Large
\scalebox{0.8}{
\begin{tabular}{ l l }
\textbf{\emph{PALABRAS CLAVE}}:	& \emph{redes de sensores inal�mbricas cognitivas}, \\ 
 				& \emph{sensores y actuadores}, \emph{placa de expansi�n}, \\
				& \emph{redes cognitivas}, \emph{redes de sensores inal�mbricas}.
\end{tabular}}
\end{table}

\endinput


% Variable local para emacs, para  que encuentre el fichero maestro de
% compilaci�n y funcionen mejor algunas teclas r�pidas de AucTeX
%%%
%%% Local Variables:
%%% mode: latex
%%% TeX-master: "../Tesis.tex"
%%% End:
