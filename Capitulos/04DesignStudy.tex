%---------------------------------------------------------------------
%
%                          Capítulo 4
%
%---------------------------------------------------------------------

\chapter{Estudio del Diseño y de la Implementación}
\label{cap:diseno}

\begin{FraseCelebre}
\begin{Frase}
El conocimiento, si no se sabe aplicar,\\
es peor que la ignorancia.
\end{Frase}
\begin{Fuente}
Charles Bukowski
\end{Fuente}
\end{FraseCelebre}

\begin{resumen}
In this chapter, the platform design is described. The design decisions, specifications, schemes and needed calculations will be exposed.
The detailed characteristics of any chosen component, (including microcontroller, radio interfaces, and power-supply options) will be provided.
\end{resumen}


%-------------------------------------------------------------------
\section{Sensores}
%-------------------------------------------------------------------
\label{diseno:sec:sensores}
%-------------------------------------------------------------------

En este apartado, se decidirán finalmente los sensores comerciales de los cuatro tipos expuestos en la sección \ref{previo:sec:sensores} que se incluirán y la forma en la que se colocarán, así como de la circuitería adicional que necesitan para funcionar. La estructura que se seguirá en los siguientes apartados se presenta a continuación.

\begin{enumerate}

\item Partiendo de lo expuesto en el capítulo \ref{cap:previo} procederemos, de manera justificada, a la elección del sensor comercial que se ajuste mejor a las necesidades del problema a resolver.
\item Se presentará un diagrama de bloques que ilustre el modo de operación interno del sensor, así como un diagrama de pines y las señales de entrada y salida del mismo.
\item Finalmente, se detallarán y se justificarán los componentes adicionales que se han de incluir al mismo.

\end{enumerate}

%-------------------------------------------------------------------
\subsection{Temperatura}
%-------------------------------------------------------------------
\label{diseno:sec:temperatura}
%-------------------------------------------------------------------

De los modelos presentados en la tabla \ref{previo:tab:temp}, decidimos escoger el { \bf Microchip MCP9800} \cite{mcp9800}, principalmente por ser el de mayor precisión con un precio asequible. Sus características detalladas se exponen a continuación.

\begin{multicols}{2}
\begin{itemize}
\item Precisión típica de $\pm$0.5ºC a 25ºC.
\item Error máximo de $\pm$1ºC entre -10ºC y 85ºC.
\item Resolución seleccionable de 9-12 bits.
\item Conexión I$^2$C.
\item Corriente de operación típica de 200 $\mu$A.
\item Corriente de apagado máxima de 1 $\mu$A.
\item Salida independiente configurable (ALERTA).
\end{itemize}
\end{multicols}

El diagrama de bloques interno del sensor de temperatura se presenta en la figura \ref{diseno:fig:temp_bloques}. Como podemos ver, se trata de un conversor \ac{ADC} de tipo sigma-delta al que luego se conecta todo el sistema de registros, el cual se encargará de transmitir la medida y de generar las alarmas.

\figuraEx{Bitmap/Capitulo4/temp_bloques}{width=.3\textwidth}{diseno:fig:temp_bloques}%
{Diagrama de bloques del MCP9800, obtenido de \cite{mcp9800}.}{Diagrama de bloques del MCP9800.}



%-------------------------------------------------------------------
\subsection{Acelerómetro}
%-------------------------------------------------------------------
\label{diseno:sec:acelerometro}
%-------------------------------------------------------------------

asd

%-------------------------------------------------------------------
\subsection{Presencia}
%-------------------------------------------------------------------
\label{diseno:sec:presencia}
%-------------------------------------------------------------------

asd

%-------------------------------------------------------------------
\subsection{Luminosidad}
%-------------------------------------------------------------------
\label{diseno:sec:presencia}
%-------------------------------------------------------------------

AsdfH VQU

%-------------------------------------------------------------------
\section{Actuadores}
%-------------------------------------------------------------------
\label{diseno:sec:actuadores}
%-------------------------------------------------------------------

ASDASDA

%-------------------------------------------------------------------
\subsection{Emisor Infrarrojo}
%-------------------------------------------------------------------
\label{diseno:sec:ir}
%-------------------------------------------------------------------

JSHDFJAD..

%-------------------------------------------------------------------
\subsection{Buzzer}
%-------------------------------------------------------------------
\label{diseno:sec:buzzer}
%-------------------------------------------------------------------


%-------------------------------------------------------------------
\subsection{LEDs}
%-------------------------------------------------------------------
\label{diseno:sec:leds}
%-------------------------------------------------------------------

XDSDFS

%-------------------------------------------------------------------
\subsection{GPIOs}
%-------------------------------------------------------------------
\label{diseno:sec:gpio}
%-------------------------------------------------------------------

SDFSDF

%-------------------------------------------------------------------
\section{Esquemático}
%-------------------------------------------------------------------
\label{diseno:sec:esquematico}
%-------------------------------------------------------------------

SDFSDF

%-------------------------------------------------------------------
\section{Resultado final}
%-------------------------------------------------------------------
\label{diseno:sec:herramientas}
%-------------------------------------------------------------------

SADFQW 


%-------------------------------------------------------------------
\subsection{Implementación en PCB}
%-------------------------------------------------------------------

SDFS 


%-------------------------------------------------------------------
\subsection{Lista de Materiales}
%-------------------------------------------------------------------

SDFSDFS

%-------------------------------------------------------------------
\subsection{Layout}
%-------------------------------------------------------------------

ÑOUIOÑUI


%-------------------------------------------------------------------
\section{Herramientas}
%-------------------------------------------------------------------
\label{diseno:sec:herramientas}
%-------------------------------------------------------------------


UYKJRTFRU7JRFU


%-------------------------------------------------------------------
\subsection{Altium Designer}
%-------------------------------------------------------------------

SDJGKFAH U

%-------------------------------------------------------------------
\subsubsection{Librería de componentes}
%-------------------------------------------------------------------


%-------------------------------------------------------------------
\subsection{Puesto de Soldadura}
%-------------------------------------------------------------------

FGHDG 

