%---------------------------------------------------------------------
%
%                          Cap�tulo 7
%
%---------------------------------------------------------------------

\chapter{Software}
\label{cap:software}

\begin{FraseCelebre}
\begin{Frase}
La gente cr�tica con el software \\
deber�a hacer su propio hardware.
\end{Frase}
\begin{Fuente}
Alan Kay
\end{Fuente}
\end{FraseCelebre}

\begin{resumen}
This chapter gives a vision about the main changes required for the firmware to be adapted to the platform, the new software implementations and the behavior of the final demo application.
\end{resumen}

%-------------------------------------------------------------------
\section{Interfaz Hardware-Software}
%-------------------------------------------------------------------
\label{software:sec:interfaz}
%-------------------------------------------------------------------

La interfaz \textit{hardware}-\textit{software} no es m�s que una capa, dentro del sistema, que se encarga de abstraer los recursos \textit{hardware} por medio de funciones que pueden ser invocadas y que rigen su funcionamiento. Es por ello que esta interfaz tambi�n se la conoce por el nombre de \ac{HAL}.

El lenguaje usado para la realizaci�n de la misma ser� C, puesto que se dispone de una amplia librer�a de funciones ya desarrolladas por el fabricante para el microcontrolador en este lenguaje, y puesto que fue el \textit{software} usado para escribir el \ac{CNGD}. El manual de referencia usado ha sido \cite{thec}. 

Nuestro c�digo ser� implementado b�sicamente en dos ficheros: \verb@SensorsHAL.c@, que ser� el fichero maestro en el que se encuentren detalladas las acciones llevadas a cabo por las funciones, y \verb@SensorsHAL.h@, en el cual se encuentran las cabeceras exportadas de las funciones, de forma que las mismas puedan ser accedidas desde otro punto del programa en el que se incluya este fichero.

Todo el c�digo desarrollado se encuentra accesible p�blicamente en Internet siguiendo el siguiente enlace o consultando el c�digo QR de la figura \ref{software:fig:qrgithub}: \url{http://github.com/enriquejcobo/tfg-soft}

\figuraEx{Bitmap/Capitulo7/qrgithub}{width=.3\textwidth}{software:fig:qrgithub}%
{Repositorio p�blico del \textit{software}.\\\url{http://github.com/enriquejcobo/tfg-soft}}{Repositorio p�blico del \textit{software}.}

%-------------------------------------------------------------------
\subsection{Funciones generales y adaptaci�n}
%-------------------------------------------------------------------
\label{software:sec:general}
%-------------------------------------------------------------------

En los siguientes subapartados iremos presentado las funciones que se han desarrollado para hacer la abstracci�n a nivel \textit{software} de nuestros sensores y actuadores. En primer lugar presentamos las funciones que son comunes al propio sistema, debido a que engloban varios dispositivos del mismo. Estas son:

\begin{itemize}
\item \verb@InitSensors@. Su misi�n es la de preparar los perif�ricos dentro del \ac{MCU}, de forma que se habiliten los necesarios y en los pines correctos.
\item \verb@IntTmp@. Rutina de atenci�n a la interrupci�n causada por el temporizador interno TMR5, necesaria tanto para definir tiempos de espera como se�ales moduladas.
\item \verb@IntCN@. Rutina de atenci�n a la interrupci�n motivada por el cambio en alguna se�al rutada hacia un \ac{CN}, tomando las acciones necesarias.
\end{itemize}

%-------------------------------------------------------------------
\subsection{Sensores}
%-------------------------------------------------------------------
\label{software:sec:sensores}
%-------------------------------------------------------------------

%-------------------------------------------------------------------
\subsubsection{Temperatura}
%-------------------------------------------------------------------
\label{software:sec:temp}
%-------------------------------------------------------------------

%-------------------------------------------------------------------
\subsubsection{Aceler�metro}
%-------------------------------------------------------------------
\label{software:sec:acc}
%-------------------------------------------------------------------

%-------------------------------------------------------------------
\subsubsection{Presencia}
%-------------------------------------------------------------------
\label{software:sec:pir}
%-------------------------------------------------------------------

%-------------------------------------------------------------------
\subsubsection{Luminosidad}
%-------------------------------------------------------------------
\label{software:sec:temp}
%-------------------------------------------------------------------

%-------------------------------------------------------------------
\subsection{Actuadores}
%-------------------------------------------------------------------
\label{software:sec:actuadores}
%-------------------------------------------------------------------

%-------------------------------------------------------------------
\subsubsection{Emisor infrarrojo}
%-------------------------------------------------------------------
\label{software:sec:ir}
%-------------------------------------------------------------------

%-------------------------------------------------------------------
\subsubsection{Buzzer}
%-------------------------------------------------------------------
\label{software:sec:buzz}
%-------------------------------------------------------------------

%-------------------------------------------------------------------
\subsubsection{LEDs}
%-------------------------------------------------------------------
\label{software:sec:leds}
%-------------------------------------------------------------------

%-------------------------------------------------------------------
\section{Demostraci�n}
%-------------------------------------------------------------------
\label{software:sec:demo}
%-------------------------------------------------------------------
\todo{Escribir...}

%-------------------------------------------------------------------
\section{Herramientas}
%-------------------------------------------------------------------
\label{software:sec:herramientas}
%-------------------------------------------------------------------

Al igual que en casos anteriores, presentamos las herramientas utilizadas para la realizaci�n del \textit{software} de tanto la \ac{HAL} como de la demo.

%-------------------------------------------------------------------
\subsection{MPLAB X}
%-------------------------------------------------------------------
\label{software:sub:mplabx}
%-------------------------------------------------------------------

MPLAB X es la \ac{IDE} desarrollada por Microchip para la programaci�n de sus microcontroladores. Est� basado en el NetBeans \ac{IDE}, \textit{open-source} de Oracle, e incluye integrado el compilador de C para sus dispositivos. Entre algunas de las opciones que ofrece destacamos:

\begin{itemize}
\item Varias herramientas de depuraci�n.
\item \textit{Parsing} y control de sintaxis en tiempo real.
\item Hiperv�nculos que permiten una navegaci�n r�pida para acceder a las declaraciones.
\end{itemize}

M�s informaci�n puede ser consultada en el manual \cite{mplabx}.

%-------------------------------------------------------------------
\subsection{Programador: ICD 3}
%-------------------------------------------------------------------
\label{software:sub:icd3}
%-------------------------------------------------------------------

El \ac{ICD} es el dispositivo que permite programar el microcontrolador. En nuestro caso usaremos el \ac{ICD} 3 \cite{icd3}, el cual ha sido desarrollado para los productos de Microchip. Como caracter�stica adicional, ester dispositivo puede ser usado, en combinaci�n con el MPLAB, para la depuraci�n en tiempo real del \textit{software} corriendo en el \ac{MCU}, permitiendo hasta 6 puntos de parada. Este hecho resultar� crucial en los siguientes apartados.


%
% Variable local para emacs, para  que encuentre el fichero maestro de
% compilaci�n y funcionen mejor algunas teclas r�pidas de AucTeX
%%%
%%% Local Variables:
%%% mode: latex
%%% TeX-master: "../Tesis.tex"
%%% End:
