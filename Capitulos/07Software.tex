%---------------------------------------------------------------------
%
%                          Cap�tulo 7
%
%---------------------------------------------------------------------

\chapter{Software}
\label{cap:software}

\begin{FraseCelebre}
\begin{Frase}
La gente cr�tica con el software \\
deber�a hacer su propio hardware.
\end{Frase}
\begin{Fuente}
Alan Kay
\end{Fuente}
\end{FraseCelebre}

\begin{resumen}
This chapter gives a vision about the main changes required for the firmware to be adapted to the platform, the new software implementations and the behavior of the final demo application.
\end{resumen}

%-------------------------------------------------------------------
\section{Interfaz Hardware-Software}
%-------------------------------------------------------------------
\label{software:sec:interfaz}
%-------------------------------------------------------------------

\todo{Escribir...}

%-------------------------------------------------------------------
\subsection{Demostraci�n}
%-------------------------------------------------------------------
\label{software:sec:demo}
%-------------------------------------------------------------------
\todo{Escribir...}

%-------------------------------------------------------------------
\section{Herramientas}
%-------------------------------------------------------------------
\label{software:sec:herramientas}
%-------------------------------------------------------------------

Al igual que en casos anteriores, presentamos las herramientas utilizadas para la realizaci�n del \textit{software} de tanto la \ac{HAL} como de la demo.

%-------------------------------------------------------------------
\subsection{MPLAB X}
%-------------------------------------------------------------------
\label{software:sub:mplabx}
%-------------------------------------------------------------------

MPLAB X es la \ac{IDE} desarrollada por Microchip para la programaci�n de sus microcontroladores. Est� basado en el NetBeans \ac{IDE}, \textit{open-source} de Oracle, e incluye integrado el compilador de C para sus dispositivos. Entre algunas de las opciones que ofrece destacamos:

\begin{itemize}
\item Varias herramientas de depuraci�n.
\item \textit{Parsing} y control de sintaxis en tiempo real.
\item Hiperv�nculos que permiten una navegaci�n r�pida para acceder a las declaraciones.
\end{itemize}

M�s informaci�n puede ser consultada en el manual \cite{mplabx}.

%-------------------------------------------------------------------
\subsection{Programador: ICD 3}
%-------------------------------------------------------------------
\label{software:sub:icd3}
%-------------------------------------------------------------------

El \ac{ICD} es el dispositivo que permite programar el microcontrolador. En nuestro caso usaremos el \ac{ICD} 3 \cite{icd3}, el cual ha sido desarrollado para los productos de Microchip. Como caracter�stica adicional, ester dispositivo puede ser usado, en combinaci�n con el MPLAB, para la depuraci�n en tiempo real del \textit{software} corriendo en el \ac{MCU}, permitiendo hasta 6 puntos de parada. Este hecho resultar� crucial en los siguientes apartados.


%
% Variable local para emacs, para  que encuentre el fichero maestro de
% compilaci�n y funcionen mejor algunas teclas r�pidas de AucTeX
%%%
%%% Local Variables:
%%% mode: latex
%%% TeX-master: "../Tesis.tex"
%%% End:
