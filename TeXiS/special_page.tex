
% Página que en la cara de detrás del folio de la portada.
% Ponemos que el documento esta maquetado con TeXiS y la
% aclaración de que debe imprimirse a doble cara.
\newpage
\thispagestyle{empty}
\mbox{ }

\begin{center}
\includegraphics[width=0.4\textwidth]{Imagenes/Bitmap/Otras/memoriaonline}
\end{center}

\begin{small}
\begin{center}
Versión online. \emph{http://goo.gl/mQ3Mez}.
\end{center}
\end{small}


\vfill%space*{4cm}
\begin{small} 
\begin{center}
\ifx\noTeXiSCreditsVal\undefined
  Documento creado a partir de \texis\ v.\texisVer.
\else
\mbox{ }
\fi
\end{center}
\end{small}
\vspace*{1cm}
\begin{small} 
\begin{center}
\ifx\explicacionDobleCaraVal\undefined
\mbox{ }
\else
\noindent Este documento está preparado para su impresión a doble cara.
\fi
\end{center}
\end{small}

\begin{small}
\begin{center}
\vspace*{1cm}
Todos los ficheros usados para hacer este documento son accesibles públicamente en el siguiente repositorio: \emph{https://github.com/agus-xyz/cNGD-mem}.
\end{center}
\end{small}

